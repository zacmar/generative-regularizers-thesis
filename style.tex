\usepackage[compact]{titlesec}
\newcommand{\flatcaps}[1]{\textsc{#1}}
\titleformat{\section}{\large\bfseries}{\thesection}{1em}{}
\titleformat{\subsection}{\large}{\thesubsection}{.6em}{\flatcaps}
\titleformat{\subsubsection}{}{\thesubsubsection}{.6em}{}{\itshape}
\titleformat{\paragraph}[runin]{\flatcaps}{\theparagraph}{0pt}{}

\titlespacing*{\section}{0pt}{2\baselineskip}{1\baselineskip}
\titlespacing*{\subsection}{0pt}{1\baselineskip}{1\baselineskip}
\titlespacing*{\subsubsection}{0pt}{8pt}{5pt}

\frenchspacing

\setmainfont[
	Ligatures=TeX,
	% TODO: Uncommenting the following line breaks compilation
	Numbers={OldStyle}, %, Proportional},
	SmallCapsFeatures={LetterSpace=3, Renderer=Basic},
]{Linux Libertine O}
\setmonofont[Scale=MatchLowercase]{DejaVu Sans Mono}
% \setmathfont[Scale=MatchUppercase]{libertinusmath-regular.otf}
\setmathfont{Asana-Math.otf}
\newlength{\alphabet}
\settowidth{\alphabet}{\normalfont abcdefghijklmnopqrstuvwxyz}
\usepackage{geometry}
\ifthenelse{\boolean{bPrintVersion}} % compile print or web version
{%
	\usepackage[
		colorlinks  = true,
		citecolor   = black,
		linkcolor   = black,
		urlcolor    = black,
		breaklinks  = true,
		pdfauthor   = \authorname,
		pdftitle    = \titlename,
		pdfsubject  = {PhD\ Thesis; Computer\ Vision; Inverse\ Problems; Generative\ Models},
		pdfkeywords = {computer\ vision; generative\ models; inverse\ problems},
		pdfcreator  = LaTeX with hyperref on archlinux,
		pdfproducer = luaLaTeX,
	]{hyperref}
	\geometry{inner=3cm,textwidth=2.7\alphabet,marginparwidth=4.2cm,marginparsep=.4cm,bottom=4cm}
	\floatsetup[widefigure]{%
		margins=hangoutside,
		facing=yes,
		capposition=bottom%
	}
	\floatsetup[widetable]{%
		margins=hangoutside,
		facing=yes,
		capposition=bottom%
	}
}{%
	\usepackage[
		colorlinks  = true,
		citecolor   = maincolor,
		linkcolor   = maincolor,
		linktocpage = true,
		urlcolor    = black,
		breaklinks  = true,
		pdfauthor   = \authorname,
		pdftitle    = \titlename,
		pdfsubject  = {PhD\ Thesis; Computer\ Vision; Inverse\ Problems; Generative\ Models},
		pdfkeywords = {computer\ vision; generative\ models; inverse\ problems},
		pdfcreator  = LaTeX with hyperref on archlinux,
		pdfproducer = luaLaTeX,
	]{hyperref}
	\geometry{twoside=false,left=3cm,textwidth=2.7\alphabet,marginparwidth=4.2cm,marginparsep=.4cm,bottom=4cm}
	% These floatrow starred figures dont respect the geometry change (if i use hangoutside, the figures protrude to the left and right like they should on twosided documents)
	% Not sure if this is a bug or if i can somehow streamline telling all of these packages
	% that i dont want twosided anything in the non-print version
	\floatsetup[widefigure]{%
		margins=hangright,
		facing=yes,
		capposition=bottom%
	}
	\floatsetup[widetable]{%
		margins=hangright,
		facing=yes,
		capposition=bottom%
	}
}

% prevent float-only pages
\renewcommand{\topfraction}{.8}
\renewcommand{\floatpagefraction}{.8}
\newcommand\xray{X-ray}

% Compact lists
\usepackage{enumitem}
\setlist{noitemsep}
% \setlist{nosep}
\setlist[1]{labelindent=\parindent} % < Usually a good idea
%% Have to remove the bold and indent labels at the edge, text will still be
%% aligned at \parindent
% \setlist[description]{font=\normalfont\flatcaps, labelindent=0pt}

\newcommand\optimal[1]{#1^{\ast}}
\newcommand\DiffOp{D}
\newcommand\Noise{\eta}
\newcommand\Forward{A}
\DeclareMathOperator{\Trace}{tr}
\DeclareMathOperator{\Determinant}{det}
\newcommand\Identity{I}
\newcommand\Expectation{\mathbb{E}}
\newcommand\Entropy{H}
\newcommand\Variance{\operatorname{Var}}
\newcommand\R{\mathbb{R}}
\newcommand\C{\mathbb{C}}
\newcommand\V{\mathbb{V}}\newcommand\Field{\mathbb{K}}
\newcommand\map[3]{#1 \colon #2 \to #3}
\newcommand\argm{\,\cdot\,}
\newcommand\tp{^\top}
\newcommand\Adjoint[1]{#1^{\ast}}
\newcommand\gteqmat{\succcurlyeq}
\newcommand\gtmat{\succcurly}\newcommand\lteqmat{\preccurlyeq}
\newcommand\ltmat{\preccurly}
\newcommand\FenchelConj[1]{#1^{\ast}}
\newcommand\Closure[1]{\overline{#1}}
\newcommand\ASet{\mathcal{S}}
\newcommand\ASetP{\mathcal{S}^\prime}
\DeclareMathOperator{\interior}{int}
% \DeclareMathOperator{\bdy}{\partial}
\newcommand\Interior[1]{\interior #1}
\newcommand\Boundary[1]{\partial #1}
\newcommand\ConstraintSet{C}
\newcommand\Simplex{\Delta}
\newcommand\DualSpace[1]{#1^{\ast}}
\newcommand\ConvexConjugate[1]{#1^{\ast}}
\newcommand\Biconjugate[1]{#1^{\ast\ast}}
\newcommand\conj[1]{\overline{#1}}
\newcommand\Optimal[1]{{#1}^\ast}
\newcommand\IndicatorFunction[1]{\delta_{#1}}
\newcommand\CharacteristicFunction[1]{\chi_{#1}}
\newcommand\inprod[2]{\langle #1, #2 \rangle}
\newcommand\ContinuousConvolution[2]{(#1 \ast #2)}
\renewcommand\mod{\,\mathrm{mod}\,}

\newcommand\InfConv[2]{#1\mathbin{\square}#2}
\newcommand\Convolved[2]{#1\mathbin{*}#2}
\newcommand\Moreau[2][\lambda]{#2_{#1}}
\newcommand\Ball[3]{\mathcal{B}_{#1}(#2, #3)}
\newcommand\Subdifferential{\partial}
\newcommand\SigmaAlgebraSymbol{\mathfrak{F}}
\newcommand\SigmaAlgebra{\(\sigma\)-algebra}
\newcommand\SigmaAlgebras{\SigmaAlgebra{}s}
\newcommand\PowerSet[1]{2^{#1}}
\newcommand\Measure{\mu}
\newcommand\Event{\mathcal{A}^\prime}
\newcommand\PreImage[1]{{#1}^{-1}}
\newcommand\composed{\circ}
\newcommand\DistributedAs{\sim}
\newcommand\DistributionFunction{F}
\newcommand\DistributionFunctionX{\DistributionFunction_\RandomVariable}
\newcommand\DensityFunction{p}
\newcommand\DensityFunctionX{\DensityFunction_\RandomVariable}
\newcommand\DensityFunctionXEstimation{\hat{\DensityFunction}_{\RandomVariable}}
\newcommand\NormalDistribution{\mathcal{N}}
\newcommand\BorelSigma[1]{\mathfrak{B}(#1)}
\newcommand\Dirac{\delta}
\newcommand\AbsolutelyContinuous{\ll}
\newcommand\DivergenceMeasure[2]{(#1\mathbin{||}#2)}
\newcommand\KullbackLeibler[2]{\DivergenceMeasure{#1}{#2}_{\mathrm{KL}}}
\newcommand\Fisher[2]{\DivergenceMeasure{#1}{#2}_{\mathrm{F}}}
\newcommand\MeasurableSpace{(\ASet, \SigmaAlgebraSymbol)}
\newcommand\MeasurableSpaceP{(\ASetP, \SigmaAlgebraSymbol^\prime)}
\newcommand\MeasureSpace{(\ASet, \SigmaAlgebraSymbol, \Measure)}
\newcommand\LebesgueMeasure{\mathfrak{L}}
\newcommand\LebesgueSpace{L}
\newcommand\RandomVariable{X}
\newcommand\StochasticProcess[1]{(X_t)_{t\in#1}}

\newcommand\ProbabilityMeasure{\mathbb{P}}
\newcommand\ProbabilityMeasureX{\ProbabilityMeasure_\RandomVariable}
\newcommand\ProbabilitySpace{(\ASet, \SigmaAlgebraSymbol, \ProbabilityMeasure)}
\DeclarePairedDelimiterX\norm[1]\lVert\rVert{
	\ifblank{#1}{\:\cdot\:}{#1}
}
\DeclarePairedDelimiterX\abs[1]{\lvert}{\rvert}{
	\ifblank{#1}{\:\cdot\:}{#1}
}
\DeclareMathOperator{\epi}{epi}
\DeclareMathOperator{\ConvexHull}{conv}
\DeclareMathOperator{\cube}{cube}
\DeclareMathOperator{\domain}{dom}
\DeclareMathOperator{\prox}{prox}
\DeclareMathOperator*{\argmin}{arg\,min}
\DeclareMathOperator*{\argmax}{arg\,max}
\DeclareMathOperator{\proj}{proj}
\DeclareMathOperator{\sign}{sign}
\DeclareMathOperator{\diag}{diag}
\DeclareMathOperator{\card}{card}
% Taken directly from the mathtools documentation, must be good
\providecommand\given{}
\newcommand\SetSymbol[1][]{%
	\nonscript\:#1\vert
	\allowbreak
	\nonscript\:
	\mathopen{}%
}
\DeclarePairedDelimiterX\Set[1]\{\}{%
	\renewcommand\given{\SetSymbol[\delimsize]}
	#1
}
\newcommand{\Hessian}{\nabla^2}
\newcommand{\Grad}{\nabla}
\newcommand{\kspace}{k-space}


\definecolor{viridisblue}{HTML}{3F4688}
\definecolor{maincolor}{HTML}{177345}
\definecolor{secondarycolor}{HTML}{731745}
\pgfplotscreateplotcyclelist{mycolors}{
	{color=MidnightBlue},
	{color=Peach},
	{color=JungleGreen},
	{color=Maroon}, 
    {color=BurntOrange}
}
\newcommand\Domain{\Omega}
\newcommand\Colorspace{F}
\newcommand\Laplace{\Delta}
\newcommand\Brownian{B}
\newcommand\PredictableSimpleProcesses{\mathcal{E}}
\newcommand\DiffusionCoefficient{f}
\newcommand\Drift{g}
\newcommand\Data{y}
\newcommand\Signal{x}
\newcommand\Fourier{F}
\newcommand\ImgDim{\R^{m\times n}}
\newcommand\ImaginaryUnit{\mathrm{i}}

\newcommand\RunningExampleFigure[3]{%
	\begin{sidefigure}
		\centering
		\begin{tikzpicture}[spy using outlines={rectangle, magnification=3, width=3.9cm, height=2cm, connect spies}]
			\node [inner sep=0, outer sep=0] at (0, 0) {\includegraphics[width=3.9cm,angle=180]{#1}};
			\spy [maincolor] on (1, -.3) in node at (0, -3);
		\end{tikzpicture}
		\caption[Running example: #2]{#3}%
		\label{fig:running example #1}
	\end{sidefigure}%
}

\newcommand\relu{\tikzexternaldisable\tikz[thick, scale=0.35]{%
		\draw (0.0, 0.0) -- ++(0.3, 0.) -- ++(0.5, 0.5);%
}\tikzexternalenable}

\pgfplotsset{colormap={inferno}{%
rgb = (1.46200e-03, 4.66000e-04, 1.38660e-02)
rgb = (2.94320e-02, 2.15030e-02, 1.14621e-01)
rgb = (9.29900e-02, 4.55830e-02, 2.34358e-01)
rgb = (1.83429e-01, 4.03290e-02, 3.54971e-01)
rgb = (2.71347e-01, 4.09220e-02, 4.11976e-01)
rgb = (3.60284e-01, 6.92470e-02, 4.31497e-01)
rgb = (4.41207e-01, 9.93380e-02, 4.31594e-01)
rgb = (5.28444e-01, 1.30341e-01, 4.18142e-01)
rgb = (6.09330e-01, 1.59474e-01, 3.93589e-01)
rgb = (6.94627e-01, 1.95021e-01, 3.54388e-01)
rgb = (7.69556e-01, 2.36077e-01, 3.07485e-01)
rgb = (8.41969e-01, 2.92933e-01, 2.48564e-01)
rgb = (8.98192e-01, 3.58911e-01, 1.88860e-01)
rgb = (9.44285e-01, 4.42772e-01, 1.20354e-01)
rgb = (9.72590e-01, 5.29798e-01, 5.33240e-02)
rgb = (9.86964e-01, 6.30485e-01, 3.09080e-02)
rgb = (9.84865e-01, 7.28427e-01, 1.20785e-01)
rgb = (9.66243e-01, 8.36191e-01, 2.61534e-01)
rgb = (9.46392e-01, 9.30761e-01, 4.42367e-01)
rgb = (9.88362e-01, 9.98364e-01, 6.44924e-01)
}}
\pgfplotsset{colormap={flare}{%
rgb = (0.92539502, 0.64345456, 0.47594352)
rgb = (0.92077582, 0.59804722, 0.44818634)
rgb = (0.9155979, 0.55210684, 0.42070204)
rgb = (0.90921368, 0.5056543, 0.39544411)
rgb = (0.90077904, 0.45884905, 0.37556121)
rgb = (0.888292, 0.40830288, 0.36223756)
rgb = (0.87199254, 0.3633634, 0.35974223)
rgb = (0.84916723, 0.32289973, 0.36711424)
rgb = (0.81942908, 0.28911553, 0.38102921)
rgb = (0.7826624, 0.26420493, 0.39754146)
rgb = (0.73695678, 0.24620072, 0.41357737)
rgb = (0.69226314, 0.23413578, 0.42480327)
rgb = (0.64795375, 0.22217149, 0.43330852)
rgb = (0.60407977, 0.21017746, 0.43913439)
rgb = (0.56041794, 0.19845221, 0.44207535)
rgb = (0.51278481, 0.18693492, 0.44112605)
rgb = (0.46818879, 0.17788392, 0.43552047)
rgb = (0.42355299, 0.16934709, 0.42581586)
rgb = (0.37928736, 0.16052483, 0.41270599)
rgb = (0.33604378, 0.15006017, 0.39835754)
}}

\newcommand{\drawcolorbar}{\tikzexternaldisable%
	\pgfplotscolorbardrawstandalone[
		scale=0.32, colormap={example}{samples of colormap = (8 of inferno)},
		colorbar horizontal,point meta max=0.2,colorbar style={ticks=none},
	]%
\tikzexternalenable}
\newcommand{\drawcolorbarbw}{\tikzexternaldisable%
	\pgfplotscolorbardrawstandalone[
		scale=0.32, colormap={example}{samples of colormap = (8 of colormap/blackwhite)},
		colorbar horizontal,point meta max=0.2,colorbar style={ticks=none},
	]%
\tikzexternalenable}

\definecolor{coolwarm1}{HTML}{3A4CC0}
\definecolor{coolwarm2}{HTML}{8DB0FE}
\definecolor{coolwarm3}{HTML}{DEDDDC}
\definecolor{coolwarm4}{HTML}{F4997A}
\definecolor{coolwarm5}{HTML}{B40326}

\newcommand{\crop}{\tikzexternaldisable\tikz[thick, scale=0.35]{%
		\draw [shorten <= -0.4](0.25, 0.25) -- ++(-0.7, 0.);%
		\fill[white] (-0.15, 0.15) rectangle (-0.35, 0.35);%
		\draw (-0.25, -0.25) -- ++(0., 0.7);%

		\draw [shorten <= -0.4] (-0.25, -0.25) -- ++(0.7, 0.);%
		\fill[overlay, white] (0.15, -0.15) rectangle (0.35, -0.35);%
		\draw [shorten >= -2](0.25, 0.25) -- ++(0., -0.5);%
}\tikzexternalenable}

\pgfplotsset{%
	marginlabels/.style={%
		ticklabel style={font=\fontsize{7pt}{8pt}\selectfont},
	},
	marginplot/.style={%
		height=.9\marginparwidth,
		width=.9\marginparwidth,
		scale only axis,
		axis y line=center,
		axis x line=middle,
		marginlabels,
	},
	cycle list={[indices of colormap={0,4,8,12,17} of flare]},
	% https://tex.stackexchange.com/questions/239794/group-style-key-does-not-work-in-pgfplotsset
	pogmdm group plot/.style={%
		group/x descriptions at=edge bottom,
		group/y descriptions at=edge left,
		group/vertical sep=3mm,
		group/horizontal sep=3mm,
		width=1cm,
		height=1cm,
		scale only axis,
		no markers,
		ticklabel style={font=\tiny},
		grid=major,
		cycle list={[indices of colormap={0,4,8,12,17} of flare]},
		thick,
	},
}
\def\wwidth{2.7cm}
\def\hpad{2mm}
\def\vpad{0mm}
% .2pt is the line width of the spy rectangle in the "unzoomed" image, see
% https://tex.stackexchange.com/questions/145853/how-can-i-change-the-line-width-of-the-spy-on-node-in-tikz-spy-library
% i dont know why we have to multiply it by 4 instead of the maginifaction, which is 3?
% i think the zoomed region simply has different line width, not exactly magnification*(linewidth of the rectangle indicating the area to be zoomed)
\pgfmathsetlengthmacro\spysize{\wwidth / 2 - 4 * .2pt}
\tikzset{
	mrispy/.style={%
		spy using outlines={%
			rectangle,
			magnification=3,
			width=\spysize,
			height=\spysize,
			maincolor,
		}
	},
	font=\footnotesize,
}
% Put footnotes into margin
\usepackage[footnote,perchapter=true]{snotez}
\usepackage{ragged2e}
\setsidenotes{%
	text-mark-format=\textsuperscript{\normalfont#1},
	% text-format+=\RaggedRight,
	note-mark-format=#1:,
	note-mark-sep=\enskip,
}


% https://tex.stackexchange.com/questions/59926/how-to-draw-brownian-motions-in-tikz-pgf
% Create a function for generating inverse normally distributed numbers using the Box–Muller transform
\pgfmathdeclarefunction{invgauss}{2}{%
  \pgfmathparse{sqrt(-2*ln(#1))*cos(deg(2*pi*#2))}%
}
% Code for brownian motion
\makeatletter
\pgfplotsset{
    table/.cd,
    brownian motion/.style={
        create on use/brown/.style={
            create col/expr accum={
                (\coordindex>0)*(
                    max(
                        min(
                            invgauss(rnd,rnd)*0.1+\pgfmathaccuma,
                            \pgfplots@brownian@max
                        ),
                        \pgfplots@brownian@min
                    )
                ) + (\coordindex<1)*\pgfplots@brownian@start
            }{\pgfplots@brownian@start}
        },
        y=brown, x expr={\coordindex/100},
        brownian motion/.cd,
        #1,
        /.cd
    },
    brownian motion/.cd,
            min/.store in=\pgfplots@brownian@min,
        min=-inf,
            max/.store in=\pgfplots@brownian@max,
            max=inf,
            start/.store in=\pgfplots@brownian@start,
        start=0
}
\makeatother
%

% Initialise an empty table with a certain number of rows
\pgfplotstablenew{201}\loadedtable % How many steps?

\newcommand\UndercompleteModel{p^{\mathrm{filt}}_\theta}
\newcommand\WaveletModel{p^{\mathrm{wave}}_\theta}
\newcommand\OvercompleteModel{p^{\mathrm{conv}}_\theta}
\newcommand\EPLL{\operatorname{epll}_\theta^{\mathrm{filt}}}
\newcommand\GenericModel{p_\theta}


\newcommand\WaveletFunction{\omega}
\newcommand\ScalingFunction{\phi}

\newcommand\DiscreteSineTransform{\mathcal{S}}
\newcommand\Mask{M}

\newcommand\Height{m}
\newcommand\Width{n}
\newcommand\PixelSize{h}
\newcommand\MaxIntensity{a}
\newcommand\Filter{f}
\newcommand\Expert{\psi}
\newcommand\NumComponents{p}
\newcommand\NumExperts{o}

\newcommand\NumFrequencies{n_f}
\newcommand\NumCoils{c}
\newcommand\CoilSensitivity{\sigma}

\newcommand\Potential{\phi}
\newcommand\Half{\tfrac{1}{2}}
